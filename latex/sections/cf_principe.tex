\subsection{Das \glqq Completely-Fair\grqq{} Prinzip}
Die wichtigste Aufgabe eines Schedulers ist die Verteilung von knappen Ressourcen unter Menge von Klienten. 
Um dies zu realisieren, wurden bereits viele Anwendungen in Theorie und Praxis entwickelt.
Eine sehr bekannte Entwicklung beruht auf der Idee, jedem Klienten ein eigenes Gewicht zu zuweisen, welche dann je nach Höhe zum Gebrauch der begrenzten Ressource berechtigt.

Diese Verfahren, welches unter anderem unter dem Namen \glqq Proportional Share Scheduling\grqq{} bekannt ist, wurde bereits vor einigen Jahrzehnten entwickelt und fand dann z.B. in einem gewichtetem Round-Robin Verfahren Anwendung. Auch unter UNIX fand dieses Verfahren schnell als Scheduler Anwendung mit Hilfe einer Priorität-Steuerung. Dieses Verfahren ist schnell und benötigt immer eine konstante Zeit zum Auswählen des nächsten Klienten. 

Die Proportional-Scheduler können die Verteilung der Ressource über zwei Wege erreichen:

\begin{enumerate}
	\item Über die Frequenz. Das heißt, wie oft bekommt ein Klient die Zuteilung. Dies wird realisiert mit Hilfe der passenden Positionen in der Warteschlange.
	\item Über die Zeit. Dafür wird das Zeitquantum, welches einem Prozess zugeordnet wird, entsprechend verlängert oder verkürzt.
\end{enumerate}

Allerdings werden damit immer nur Klienten mit den höchsten Priorität bevorzugt und es können Klienten \glqq verhungern\grqq{}.
Damit begann die Entwicklung der sogenannten \glqq Fair Scheduler\grqq{}. Diese sollen das Problem der ungleichmäßigen bzw. ungerechten Verteilung der Prioritätssteuerung beheben und somit eine \glqq faire\grqq{} Verteilung ermöglichen. 

Eine Definition zur perfekten Fairness, welche die Grundlage der Fair-Scheduler bildet wird nach \cite{pdf Frau Ma} wie folgt definiert:

\begin{equation}
W_A(t_1,t_2) = (t_2 - t_1) \frac{S_A}{\sum_i S_i}
\label{eq:perfect_fairness}
\end{equation}

Hierbei ist $S_A$ der proportionale Anteil des Klienten $A$, $S_i$ die Anteile aller aktiven Klienten und $W_A$ stellt als Ergebnis die Menge der erhaltenen Zuweisung des Klienten $A$ in der Zeit zwischen $t1$ und $t2$ dar.

Im idealen System, wenn alle Klienten ihre Zuweisung gleichzeitig erhalten und ihren geforderten Ressourcenanspruch simultan verbrauchen könnten, wäre es möglich die obige Gleichung aufrecht zu erhalten.
Damit wäre eine "completely-fair" Verteilung gegeben. Allerdings ist dies in der Praxis nicht möglich, da immer nur ein Klienten für die knappe Ressource die Zuteilung bekommen kann.
Um eine Bewertung der aktuellen Strategie zur idealen "completely-fair" Strategie zu erhalten, kann ein "service time error" mit folgender Formel berechnet werden:

\begin{equation}
E_A(t_1,t_2) = W_A(t_1,t_2) - (t_2 - t_1) \frac{S_A}{\sum_i S_i}
\label{eq:perfect_fairness}
\end{equation}

Diese Differenz gibt den Abstand vom gegebenen Algorithmus zum idealen Algorithmus an.
Ist der Abstand positiv, so bekommt der aktuelle Klient mehr Zuteilung als er mit dem idealen Algorithmus bekommen würde. Ist der Abstand negative, so erhält der Klient entsprechend weniger.

Ziel in einer praktischen Lösung im Bezug auf "completely fair" soll es sein, diesen Abstand zu minimieren.
 



