\subsection{Funktionsweise}
Das Ziel eines \glqq Completely-Fair-Schedulers\grqq{} soll es sein, eine gute Annäherung an das Verhalten eines perfekten Multitasking-Prozessor zu errreichen.

Ein perfekter Multi\-tasking-Prozessor würde jedem Prozess immer ein perfektes $1/n$ Verhältnis zuweisen, wobei $n$ die Menge aller Prozesse darstellt. 
Bei nur zwei laufenden Prozessen heißt das, dass beide Prozesse zeitgleich mit jeweils 50 Prozent der Prozessorressource arbeiten dürfen. Dies ist in der Praxis allerdings nicht möglich, da jeder Prozessor immer nur eine Aufgabe zu einem Zeitpunkt bearbeiten kann.

Unter Linux und UNIX-artigen System werden den beiden Prozessen jeweils Zeitscheiben zugeordnet, und es findet eine sequentielle Abarbeitung statt. In diesem Fall werden die Prozesse abwechselnd mit jeweils 100 Prozent Prozessorzuteilung abgearbeitet.

Um mit diesem Prinzip eine Annäherung an einen perfekten Multitasking-Prozessor zu erreichen, müsste man die Zuteilung für jeden Prozess in unendlich kleine Zeitscheiben zerlegen und alternierend dem Prozessor zuweisen. Würde man an dieser Stelle eine Messung der Zuteilung durchführen, sähe es tatsächlich so aus, als würde alle Prozesse ihre benötigte Zeit gleichmäßig verbrauchen.
Allerdings hat diese Methode eine erheblichen Nachteil. Ein einziger Umschalt\-vorgang be\-nötigt selbst z.B. durch Ein- und Auslagerung der verschieden Daten im Speicher, wieder einen großen Teil der Prozessorleistung. Dadurch würde ein sehr schlechtes Verwaltungs- zu Verarbeitungsverhältnis entstehen und damit wäre der Scheduler sehr uneffizient.

Mit dem CF-Scheduler muss nun eine Umgebung geschaffen werden, welche stark an der des idealen Multitasking-Prozessor anlehnt und wiederum aber auch eine Effiziente Methode zur Ausnutzung der Prozessorzuteilung anbietet.
Anders als bei den prioritätsbestimmten Scheduler, welche über Zeitscheiben und Warteschlangen die Prioritäten verwalten, berechnet der CF-Scheduler die Zuteilung im Bezug zu allen anderen Prozessen welche sich im Wartemodus befinden. Der \glqq NICE\grqq-Wert, welcher in bereits in älteren Scheduler\-ver\-fahren für die Erstellung der Zeitscheiben zuständig war, wird jetzt verwendet, um den Prozessen ein Gewicht zuzuordnen. Ist der \glqq NICE\grqq-Wert hoch, was eine niedrige Priorität bedeutet, so bekommt der Prozess auch ein niedriges Gewicht zugeteilt. Andersherum bekommt der Prozess ein hohes Gewicht, wenn der \glqq NICE\grqq-Wert niedrig ist.

Mit Hilfe dieses Gewichtes wird dann das \glqq Completely-Fair\grqq-Prinzip [s. Kapitel \ref{s:fair}] angewendet. Jeder Prozess erhält eine Zeitscheibe, welche proportional zum eigenen Gewicht im Verhältnis zu den Gewichten aller Prozesse ist. Die Wahl des nächstes Prozesses fällt immer auf den Prozess, welcher bis zu diesem Zeitpunkt die geringste Zuteilung des Prozessors erhalten hat.
Zur Berechnung einer Zeitscheibe, setzt der CF-Scheduler eine sogenannte „targeted latency“. Diese stellt einen Richtwert für die verfügbare Zeit aller anstehenden Prozesse dar, welcher äquivalent im perfekten Multitasking erreicht würde.

Wenn zum Beispiel eine „targeted latency“ von 20ms errechnet wird und zwei Prozesse mit dem gleichen \glqq NICE\grqq-Wert in der Warteschlange stehen, wird jedem Prozess eine Zuteilung von 10ms gewährt.

Ein Problem ergibt sich bei dieser Anwendung, wenn die Anzahl der wartenden Prozesse gegen unendlich läuft. Damit würde die zugewiesene Zeit gegen Null laufen und es würde kein Prozess mehr die Berechtigung einer Zuteilung erhalten. Um dieses Problem zu umgehen, wird eine minimale Granulität eingeführt ,welche normalerweise mit einer Millisekunde gewählt wird. Das heißt jeder Prozess erhält immer mindestens eine Zuteilung von einer Millisekunde, egal wie groß die Menge der anstehenden Prozesse zu diesem Zeitpunkt ist. Dadurch wird allerdings das „faire“ Verhalten des Schedulers beeinträchtigt. 

Auch findet eine weitere Unterscheidung zur Nutzung der \glqq NICE\grqq-Wert statt. Wurde zuvor der absolute \glqq NICE\grqq-Wert für eine Prioritätsermittlung benutzt, so spielt jetzt nur noch der relative Abstand zum nächsten \glqq NICE\grqq-Wert eine Rolle.

Ein passendes Beispiel ist in \cite{rlove} illustriert. Gegeben ist ein Zustand mit zwei Prozessen. Ein Prozess hat den \glqq NICE\grqq-Wert 0, der andere den \glqq NICE\grqq-Wert 5. Damit würde sich eine Prozessorzuteilzeit von 15ms für den Prozess mit \glqq NICE\grqq-Wert 0 und eine Zuteilzeit von 5ms für den Prozess mit \glqq NICE\grqq-Wert 5 ergeben.
Im Vergleich dazu hätte man einen weiteren Zustand mit wiederum 2 Prozessen, aber dieses mal mit den \glqq NICE\grqq-Werten 10 und 15. Hier würde das selbe Ergebnis mit 15ms und 5ms zustande kommen. Damit wird gezeigt, dass die Höhe des \glqq NICE\grqq-Wertes nur noch die geometrische Differenz ändert.

Anhand dieser Erläuterung kann gesehen werden, dass der CF-Scheduler nur eine Annäherung an ein perfektes faires Scheduling erreicht. Das große Problem der CF-Scheduler ist eine übermäßige Menge an Prozessen. Wird diese Menge in einem gewissen Rahmen gehalten, so kann der CF-Scheduler das \glqq completely-fair\grqq -Prinzip gut erreichen. 
