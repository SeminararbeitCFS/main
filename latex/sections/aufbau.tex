\subsection{Funktionsweise}
Das Ziel eines \glqq Completely-Fair-Schedulers\grqq{} soll es sein, eine gute Annäherung an das Verhalten eines perfekten Multitasking-Prozessor zu errreichen.

Ein perfekter Multi\-tasking-Prozessor würde jedem Prozess immer ein perfektes 1/n Verhältnis zuweisen. 
Bei nur zwei laufenden Prozessen hieße das, das beide Prozesse zeitgleich mit jeweils 50 Prozent Prozessorzuteilung arbeiten dürfte. Dies ist in der Praxis allerdings nicht möglich, da jeder Prozessor immer nur eine Aufgabe zu einem Zeitpunkt bearbeiten kann.

Unter UNIX-artigen System werden den beiden Prozessen jeweils Zeitscheiben zugeordnet, und es findet eine sequentielle Abarbeitung statt. In diesem Fall werden die Prozesse abwechselnd mit jeweils 100 Prozent Prozessorzuteilung abgearbeitet.

Um mit diesem Prinzip eine Annäherung an einen perfekten Multitasking-Prozessor zu erreichen, müsste man die Zuteilung für jeden Prozess in unendlich kleine Zeitscheiben zerlegen und alternierend dem Prozessor zuweisen. Würde man an dieser Stelle eine Messung der Zuteilung durchführen, sähe es tatsächlich so aus, als würde alle Prozesse ihre benötigte Zeit gleichmäßig verbrauchen.
Allerdings hat diese Methode eine erheblichen Nachteil. Ein einziger Umschalt\-vorgang be\-nötigt selbst z.B. durch Ein- und Auslagerung der verschieden Daten im Speicher, wieder einen großen Teil der Prozessorleistung. Dadurch würde ein sehr schlechtes Verwaltungs- zu Verarbeitungsverhältnis entstehen und damit wäre der Scheduler sehr uneffizient.

Mit dem CF-Scheduler muss nun eine Umgebung geschaffen werden, welche stark an der des idealen Multitasking-Prozessor anlehnt und wiederum aber auch eine Effiziente Methode zur Ausnutzung der Prozessorzuteilung bereitstellt.
Anders als bei den Prioritätsbestimmten Scheduler, welche über TimeSlices die Prioritäten verwalten, berechnet der CF-Scheduler die Zuteilung im Bezug zu allen anderen Prozessen in der Warteschlange. Der nice-Wert, welcher in älteren Schedul-Ver\-fahren für die Erstellung der TimeSlices zuständig war, wird jetzt verwendet, um den Prozessen ein Gewicht zuzuordnen. Ist der NICE-Wert hoch, also eine niedrige Priorität, so bekommt der Prozess auch ein niedriges Gewicht zugeteilt. Andersherum bekommt der Prozess ein hohes Gewicht, wenn der NICE-Wert niedrig ist.

Mit Hilfe dieses Gewichtes wird dann das Completely-Fair-Prinzip angewendet. Jeder Prozess erhält ein „Zeitschlitz“, welcher proportional zu seinem eigenen Gewicht im Verhältnis zu den Gewichten aller Prozesse ist. Die Wahl des nächstes Prozesses wird fällt immer auf den Prozess, welcher bis zu diesem Zeitpunkt die geringste Zuteilung des Prozessors erhalten hat.
Zur Berechnung eines Zeitschlitzes, setzt der CF-Scheduler eine sogenannte „targeted latency“. Diese stellt einen Richtwert für die verfügbare Zeit aller anstehenden Prozesse dar, welcher äquivalent im perfekten Multitasking erreicht würde.

Wenn zum Beispiel eine „targeted latency“ von 20ms errechnet wird und zwei Prozesse mit dem gleichen NICE-Wert in der Warteschlange stehen, wird jedem Prozess eine Zuteilung von 10ms gewährt.

Ein Problem, welches bei dieser Anwendung zu Problemen führt, ist eine gegen unendlich laufende Anzahl von Prozessen, welche verwaltet werden sollen. Damit würde die Zuteilung und die zugewiesene Zeit gegen Null laufen. Um dieses Problem zu umgehen, wird eine minimale Granulität eingeführt ,welche normalerweise 1ms besitzt. D.h. jeder Prozess erhält immer mindestens eine Zuteilung von einer Millisekunde, egal wie groß die Menge der anstehenden Prozesse zu diesem Zeitpunkt ist. Damit wird allerdings das „faire“ Verhalten des Schedulers unterbrochen. Damit wird gezeigt, dass der CF-Scheduler nur annähernd faires Verhalten zeigen kann, wenn die Menge der aller Prozesse in der Warteschlange multipliziert mit 1 Millisekunde kleiner gleich der errechneten „targeted latency“ Zeit bleibt.

Eine weitere Unterscheidung zur Nutzung der NICE-Werte wird ebenfalls erreicht. Wurde zuvor der absolute NICE-Wert für eine Prioritätsermittlung benutzt, so spielt jetzt nur noch der relative Abstand zum nächsten NICE-Wert eine Rolle.

Ein Beispiel von [\cite{rlove}] wäre ein Zustand mit 2 Prozessen. Ein Zustand hat den NICE-Wert 0, der andere den NICE-Wert 5. Damit würde sich eine Prozessorzuteilzeit von 15ms für den Prozess mit NICE = 0 und eine Zuteilzeit von 5ms für den Prozess mit NICE = 5 ergeben.
Im Vergleich dazu hätte man dann einen weiteren Zustand mit wiederum 2 Prozessen, aber dieses mal mit den NICE-Werten 10 und 15. Hier würde das selbe Ergebnis mit 15ms und 5ms zustande kommen. Damit wird gezeigt, dass die Höhe des NICE-Wertes nur noch die geometrische Differenz ändert.

Anhand dieser Erläuterung kann gesehen werden, dass der CF-Scheduler nur eine Annäherung an ein perfektes faires Scheduling erreicht. Das große Problem der CF-Scheduler ist eine übermäßige Menge an Prozessen. Wird diese Menge in einem gewissen Rahmen gehalten, so kann der CF-Scheduler das \glqq completely-fair\grqq -Prinzip gut erreichen. 
