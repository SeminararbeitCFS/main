\subsection{Funktionsweise}\label{s:cfs_fktweise}

%Das Ziel eines  Completely-Fair-Schedulers{} soll es sein, eine gute Annäherung an das Verhalten eines perfekten Multitasking-Prozessor zu errreichen.
Das Ziel eines  Completely-Fair-Schedulers{} soll es sein, eine faire Aufteilung der Ressource Prozessor unter einer Menge von ausführbaren Prozessen zu erreichen.


%Ein perfekter Multi\-tasking-Prozessor würde jedem Prozess immer ein perfektes $1/n$ Verhältnis zuweisen, wobei $n$ die Menge aller Prozesse darstellt."
Im einfachsten Fall, wenn keine Prioritäten unter den Prozessen vergeben werden, ist eine perfekte Aufteilung der Ressource dann gegeben, wenn zu jedem Zeitpunkt jeder Prozess immer eine Zuteilung im Verhältnis $1/n$ erhält, wobei $n$ die Menge aller Prozesse darstellt.
Bei zwei laufenden Prozessen heißt das, dass beide Prozesse zeitgleich mit jeweils 50 Prozent der Prozessorressource arbeiten. Dies ist in der Praxis allerdings nicht möglich, da jeder Prozessor immer nur einen Prozess zu einem Zeitpunkt ausführen kann.

Unter Linux und unix\-artigen Systemen werden daher den beiden Prozessen Zeitscheiben zugeordnet und es findet eine sequentielle Abarbeitung statt. Anschließend werden die Prozesse abwechselnd mit jeweils der maximal zur Ver\-fügung stehenden Prozessorkapazität entsprechend ihrer Zeitscheibe verarbeitet.

%Um mit diesem Prinzip eine Annäherung an einen perfekten Multitasking-Prozessor zu erreichen, müsste man die Zuteilung für jeden Prozess in unendlich kleine Zeitscheiben zerlegen und alternierend dem Prozessor zuweisen.
Je kleiner die Zeitscheiben gewählt werden, desto mehr wird ein nahezu zeitgleiches Abarbeiten der anstehenden Prozesse erreicht.
Würde bei einem Schedule-Verfahren mit winzig kleinen Zeitscheiben zu einer beliebigen Zeit eine Messung der bereits zugeteilten Anteile der Prozesse durchgeführt, so sähe es tatsächlich so aus als wür\-den alle Prozesse ihr Quantum gleichzeitig verbrauchen.
Allerdings hätte diese Methode einen erheblichen Nachteil: Ein einziger Umschalt\-vorgang be\-nötigt selbst z.B. durch Ein- und Auslagerung der verschiedenen Daten im Speicher, einen großen Teil der Prozessorleistung. Dadurch entsteht ein sehr schlechtes Verwaltungs- zu Verarbeitungs\-verhält\-nis und somit ist der Scheduler sehr uneffizient. Damit es nicht zu dieser Problematik beim CFS-Scheduler kommen kann, wird eine minimale Zeitscheibe festgelegt. Diese minimale Zeitscheibe wird im Kernel als \textit{sysctl\_sched\_\\min\_granu\-larity} bezeichnet \cite{paperfairness}.
Somit wird sichergestellt, dass der Verwaltungsaufwand zum Umschalten der Prozesse die zugeteilte Bearbeitungszeit der Prozesse nicht übersteigt.


%Mit dem CFS-Scheduler muss nun eine Umgebung geschaffen werden, welche stark an der des idealen Multitasking-Prozessor anlehnt und wiederum aber auch die Effizienz zur Ausnutzung der Prozessorressource in einen akzeptablen Bereich bewegt.
Anders als bei den prioritätsbasierenden Scheduler, welche über Zeitscheiben und Warte\-schlan\-gen die Prioritäten verwalten, berechnet der CFS-Scheduler die Zuteilung im Bezug zu allen anderen Prozessen, welche sich in der entsprechenden Warteschlange des Prozessors befinden. Der  \textit{nice}-Wert, welcher bereits in älteren Scheduler-\-Ver\-fahren für die Erstellung der Zeitscheiben zuständig war, wird jetzt verwendet, um den Prozessen ein Gewicht zuzuordnen. Ist der \textit{nice}-Wert hoch, was eine niedrige Priorität bedeutet, so bekommt der Prozess auch ein niedriges Gewicht zugeteilt. Andersherum bekommt der Prozess ein hohes Gewicht, wenn der  \textit{nice}-Wert niedrig ist.

Mit Hilfe dieses Gewichtes wird dann versucht, dass \textit{com\-pletely-fair} Prinzip [Kapitel \ref{s:fair}] zu erreichen. 

Analog der Formel (\ref{eq:perfect_fairness}) wird zunächst mit Hilfe des ermittelten Gewichtes und der zu berechnenden Periode eine ideale Zeitscheibe errechnet: %Dies geschieht nach \cite{paperfairness} mit folgender Formel:  

\begin{equation}
slice_{i} = \frac{weight_{i}}{\sum_{i=1}^{n} weight_{i}} * period
\label{eq:slice}
\end{equation}
%umme_{i=1}^{n} w %eight_i


%\begin{equation}
%slice = \frac{se\rightarrow load.weight}{cfs\_rq\rightarrow load.weight} * period
%\label{eq:slice}
%\end{equation}
Hierbei ist \textit{$weight_{i}$} das Gewicht des Prozesses aus dem entsprechendem \textit{nice}-Wert, welches durch ein von dem Kernel bereit gestelltes Array \texttt{prio\_to\_weight[]} bestimmt wird. In \cite{nikita} wird der Inhalt des Arrays mit folgenden Werten angegeben:

%Hierbei ist \textit{se$\rightarrow$load.weight} das Gewicht des Prozesses aus dem entsprechendem \textit{nice}-Wert, welches durch ein von dem Kernel bereit gestelltes Array \texttt{prio\_to\_weight[]} bestimmt wird. In \cite{nikita} wird der Inhalt des Arrays mit folgenden Werten angegeben:

\begin {table}[h]
\begin{center}
\begin{tabular}{r|rrrrr}
 &	\multicolumn{5}{c}{\textit{Gewichte}} \\
	\hline\hline
	\\[\dimexpr-\normalbaselineskip+2pt]
	\textit{nice} & &+1	&	+2	& +3	& +4	\\
	\hline
    \\[\dimexpr-\normalbaselineskip+2pt]
	-20	&	88761	&	71755	&	56483	&	46273	&	36291	\\
	-15	&	29154 	&	23254 	&	18705 	&	14949 	&	11916	\\
	-10	&	9548	&	7620	&	6100	&	4904	&	3906	\\
	-5 	&	3121 	&	2501 	&	1991 	&	1586 	&	1277	\\
	0	&	1024 	&	820		&	655		&	526		&	423		\\
	5 	&	335 	&	272		& 	215		&	172		&	137		\\
	10	&	110		&	87 		&	70 		&	56 		&	45		\\
	15 	&	36 		&	29 		&	23 		&	18 		&	15	 	\\		
\end{tabular}
\caption {\textit{nice}-Wert zu Gewicht} \label{tab:nice2weight} 
\end{center}
\end{table}


In dieser Tabelle sind die entsprechenden Zuweisungen vom \textit{nice}-Wert zum Gewicht dargestellt. 
Dabei sind alle vierzig \textit{nice}-Werte mit entsprechenden Gewichten der Reihe nach absteigend gezeigt.
Spalte eins und Zeile zwei dienen dabei als Hilfsindex zum Ablesen. Das passende Gewicht zum \textit{nice}-Wert -18, kann z.B. in der Zeile mit \textit{nice}-Wert -20 und der Spalte +2 (Rechnung: -20+2=-18) mit einem Wert von 56483 ermittelt werden.\\
Die angegebenen Gewichte ensprechen ungefähr der Rechnung $1024/1.25^{nice}$.
%Die Abstände zwischen den Zahlen sind so gewählt, dass bei einem Auf- bzw. Absteigen eines \textit{nice}-Wertes die CPU-Zuteilungszeit um 10\% erhöht bzw. um 10\% erniedrigt wird. Damit ergibt sich ein Faktor zwischen nebeneinander liegenden Gewichten von ca. 1,25. Dies kann Quellcode der Linux-Kernels unter \texttt{kernel/sched.c} an entsprechender Stelle eingesehen werden. 

Der Ausdruck \textit{$\sum_{i=1}^{n} weight_{i}$} ist die Summe aller Gewichte der momentan aktiven Prozesse der Warte\-schlan\-ge.
%Die Variable \textit{cfs\_rq$\rightarrow$load.weight} ist die Summe aller Gewichte der momentan aktiven Prozesse der Warte\-schlan\-ge.

\textit{period} ist ein Zeitabschnitt, für welchen die Prozesse ihren entsprechenden proportionalen Zeitanteil erhalten sollen. Dieser Wert ist einstellbar.
Die maximal mögliche Anzahl von Prozessen für diesen Zeitabschnitt kann z.B. mit der Division des Wertes \textit{period} der aktuellen Periode durch den Wert der kleinst mög\-li\-chen Zeitscheibe, enthalten in der Variablen \texttt{sys\-ctl\_sched\_min\_granulari\-ty}, ermittelt werden.  
%Errechnet man z.B. die maximal gültige Anzahl von Prozessen für diesen Zeitabschnitt mit $period %/   und 
Über\-steigt die Anzahl der aktuellen Prozesse diesen Wert, so muss \textit{period} entsprechend angepasst werden.


%Erreicht z.B. die Anzahl der lauf\-fähigen Prozesse eine bestimmte Grenze, so muss der Wert von \textit{period} erhöht werden. 

%Die Grenze ergibt sich aus dem aktuellen Abschnitt und des festgelegten Wertes von \textit{sysctl\_sched\_min\_granularity}. 

%Diese Grenze wird im Kernel mit Hilfe der Variable \textit{nr\_latency} ermittelt. Diese Variable setzt sich aus dem Verhätlnis von \textit{period} und der minmalen Zeitscheibe für einen Prozess zusammen \cite{paperfairness}.

%\begin{equation}
%nr\_latency = \frac{sysctl\_sched\_latency}{sysctl\_sched\_min\_granularity}
%\label{eq:nr_latency}
%\end{equation}

%Somit gibt die Variable \textit{nr\_latency} die maximal mögliche Anzahl von Prozessen mit gegebenen Parametern vor. Über\-steigt nun die Anzahl der aktuell lauffähigen Prozesse diesen Wert, so muss der Wert von \textit{period}, bzw. in Formel (\ref{eq:nr_latency}) ensprechend \textit{sysctl\_sched\_latency}, erhöht werden.

Im anschließenden Schritt soll eine faire Auswahl auf den nächsten ausführbaren Prozess getroffen werden. Dies geschieht mit einer weiteren Bewertung der Prozesse im Bezug zu ihrer bereits ausgeführten Zeit. Dazu dient im CFS-Algorithmus die Variable \textit{virtual runtime}. Nach \cite{usenix} ist dieser Wert der Quotient der erhaltenen CPU-Zeit $W_{i}$ zu einem Zeitpunkt $t$ und dem ermittelten Gewicht \textit{$weight_{i}$} des Prozesses $i$.
%dem proportionalen Gewicht $S_A$ aus Formel (\ref{eq:perfect_fairness}):
\begin{equation}
virt. \ runtime(t) = \frac{W_{i}(t)}{weight_{i}}
\label{eq:vruntime}
\end{equation}
Mit diesem Verhältnis wird ein Wert geschaffen, der für alle aktiven Prozesse nach Ablauf der aktuellen Periode annäherungs\-weise gleich hoch ist.\\
Unter der Annahme, dass ein Prozess i die CPU-Zeit $W_{i}$ im Zeitinterval $[0,t]$ erhalten hat und $W_{i}$ die perfekte Fairness erfüllt, gilt:
\begin{equation}
W_{i}(t) = \frac{weight_{i}}{\sum_{j=1}^{n} weight_j} * t
\label{eq:cputime}
\end{equation}
Formel (\ref{eq:cputime}) eingesetzt in Formel (\ref{eq:vruntime}) ergibt nach Kürzen:
\begin{equation}
virt. \ runtime(t) = \frac{t}{\sum_{j=1}^{n} weight_j}
\label{eq:vruntime2}
\end{equation}
Bei Betrachtung der rechten Seite des Terms ist zu erkennen, dass die \textit{virtual runtime} für die Dauer der Ausführungs"-zeit für jeden Prozess ständig steigend ist \cite{rlove}. Ebenso ist die \textit{virtual runtime} nach Zeit $t$ für jeden Prozess gleich, es erreichen alle Prozesse die perfekte Fairness. 
Je kleiner der Wert für \textit{virtual runtime} ist, desto höher ist die Dringlichkeit dieses Prozesses den Prozessor für sich zugewiesen zu bekommen. 
Wird immer der Prozess mit der niedrigsten \textit{virtual runtime} gewählt, so kann auch eine ständige An\-nähe\-rung an das \textit{completely-fair} Prinzip garantiert werden. 
Damit ist möglich, dass \textit{completely-fairness} Prinzip trotz unterschiedlicher Gewichte und \textit{slices} einzuhalten.
%Ferner beschreibt \cite{mjones} die Funktionalität von CFS, dass eine gewisse \textit{Sleeper-Fairness} implementiert ist: Prozesse, die zu einem Zeitpunkt nicht lauffähig sind (z.B. beim Warten auf I/O"--Operationen) erhalten ei\-nen vergleichbaren Anteil am Prozessor, wenn sie diesen benö"-tigen.\\
%Nach \cite{paperfairness} wird \textit{virtual runtime} mit folgender Formel ermittelt:
%\begin{multline}
%virt. runtime_{i} \mathrel{+}= \frac{weight_{0}}{weight_{i}} * runtime(i,t) \\ = \frac{period}{\sum_{i=1}^{n} weight_{i}} * weight_{0}
%\label{eq:vruntime}
%\end{multline}
%\begin{multline}
%vruntime \mathrel{+}= \frac{delta\_exec}{se\rightarrow load.weight} * NICE\_0\_LOAD \\ = \frac{period}{cfs\_rq\rightarrow load.weight} * NICE\_0\_LOAD
%\label{eq:vruntime}
%\end{multline}

%Hierbei stellt $virt. runtime_{i}$ die bereits ausgeführte Zeit des jeweiligen Prozesses dar und \textit{$weight_{0}$} ist der Einheitswert der Gewichte, welcher das Gewicht des \textit{nice}-Wertes 0 darstellt.  
%Hierbei stellt \textit{delta\_exec} die bereits ausgeführte Zeit der jeweiligen Prozesses dar und \textit{NICE\_0\_LOAD} ist der Einheitswert der Gewichte, welcher das Gewicht des \textit{nice}-Wertes 0 darstellt.  
%welcher aus einer statischen 10-bit Dualzahl besteht $(2^{10}=1024)$ \cite{paperfairness}. 
%Mit Hilfe des ersten Terms aus Formel (\ref{eq:vruntime}) wird die aktuelle virtuellen Laufzeit des jeweiligen Prozesses bestimmt. Der zweite Term dient zur Bestimmung der allgemeinen virtuelle Laufzeit, welche jeder Prozess nach Ablauf der Periodenzeit \textit{period} für diesen Zeitabschnitt erreicht haben soll. 
%Dieser Wert soll nach Ablauf der Periodenzeit für alle Prozess gleich sein.

Mit der \textit{virtual runtime} können nun die Prozesse entsprechend ihrer Laufzeit und ihrem Gewicht bewertet und eine Rangfolge erstellt werden. Dazu wird an dieser Stelle der Rot-Schwarz-Baum [s.Kapitel \ref{s:rb_tree}] verwendet. Die Prozesse werden nach ihrem \textit{virtual runtime} Wert in den Baum einsortiert. Dabei steht der Prozess mit der kleinsten \textit{virtual runtime} am weitesten links und kann nun durch den Scheduler ohne lange Suchzeiten ermittelt und als nächster auszuführender Prozess ausgewählt werden. 


Stehen zum Beispiel drei lauffähige Prozesse  mit \textit{nice}-Werten von  $0$, $1$ und $2$ und entsprechenden Gewichten von $1024$, $820$ und $655$ in einer Warteschlange für einen Prozessor, dann ergeben sich die idealen Zeitscheiben bei einer Periode von $20ms$ mit Formel (\ref{eq:slice}) gerundet zu $8.2ms$, $6.6ms$ und $5.2ms$ und in Summe genau $20ms$. Für die \textit{virtual runtime} nach $t=20ms$ ergibt sich also mit Formel (\ref{eq:vruntime2}) ein Wert von $20ms/(1024+820+655)=0.008ms$.



Damit wird gezeigt, dass der \textit{virtual runtime} Wert dazu verwendet werden kann, eine gerechte Aufteilung entsprechend der \textit{nice}-Werte zu erreichen. 
Sind alle \textit{virtual runtime} Werte nach Abschluss der Periode gleich, so kann mit einer Annäherung an das \textit{completely-fair} gerechnet werden.


%Ermittelt man nun dazu passend die \textit{virtual runtime}-Werte mit Formel (\ref{eq:runtime}) und zwar so, dass jeder Prozess seine Zuteilung für diese Periode vollständig ausgenutzt hätte, so wird ersichtlich, dass alle Prozesse nun den gleichen \textit{virtual runtime}-Wert erreicht haben.

 %von ca. $8*10^{-6}$ besitzen. Damit wird eine faire Aufteilung mit Be\-rück\-sichtigung der \textit{Prioritäten} gewährt.

%Jeder Prozess er\-hält eine Zeitscheibe, welche proportional zum eigenen Gewicht im Verhältnis zu den Gewichten aller Prozesse ist. Die Wahl des nächstes Prozesses fällt immer auf den Prozess, welcher bis zu diesem Zeitpunkt die geringste Zuteilung des Prozessors erhalten hat.
%Zur Berechnung einer Zeitscheibe, setzt der CFS-Scheduler eine sogenannte „targeted latency“. Diese stellt einen Richtwert für die verfügbare Zeit aller anstehenden Prozesse dar, welcher äquivalent im perfekten Multitasking erreicht würde.

%Wenn zum Beispiel eine „targeted latency“ von 20ms errechnet wird und zwei Prozesse mit dem gleichen  \textit{nice}-Wert in der Warteschlange stehen, wird jedem Prozess eine Zuteilung von 10ms gewährt.

%Ein Problem ergibt sich bei dieser Anwendung, wenn die Anzahl der wartenden Prozesse gegen unendlich läuft. Damit würde die zugewiesene Zeit gegen Null laufen und es würde kein Prozess mehr die Berechtigung einer Zuteilung erhalten. Um dieses Problem zu umgehen, wird eine minimale Granulität eingeführt ,welche normalerweise mit einer Millisekunde gewählt wird. Das heißt jeder Prozess erhält immer mindestens eine Zuteilung von einer Millisekunde, egal wie groß die Menge der anstehenden Prozesse zu diesem Zeitpunkt ist. Dadurch wird allerdings das „faire“ Verhalten des Schedulers beeinträchtigt. 

%Auch findet eine weitere Unterscheidung zur Nutzung der  \textit{nice}-Wert statt. Wurde zuvor der absolute  \textit{nice}-Wert für eine Prioritätsermittlung benutzt, so spielt jetzt nur noch der relative Abstand zum nächsten  \textit{nice}-Wert eine Rolle.

%Ein passendes Beispiel ist in \cite{rlove} illustriert. Gegeben ist ein Zustand mit zwei Prozessen. Ein Prozess hat den  \textit{nice}-Wert 0, der andere den  \textit{nice}-Wert 5. Damit würde sich eine Prozessorzuteilzeit von 15ms für den Prozess mit  \textit{nice}-Wert 0 und eine Zuteilzeit von 5ms für den Prozess mit  \textit{nice}-Wert 5 ergeben.
%Im Vergleich dazu hätte man einen weiteren Zustand mit wiederum 2 Prozessen, aber dieses mal mit den  \textit{nice}-Werten 10 und 15. Hier würde das selbe Ergebnis mit 15ms und 5ms zustande kommen. Damit wird gezeigt, dass die Höhe des  \textit{nice}-Wertes nur noch die geometrische Differenz ändert.

%Anhand dieser Erläuterung kann gesehen werden, dass der CFS-Scheduler nur eine Annäherung an ein perfektes faires Scheduling erreicht. Das große Problem der CFS-Scheduler ist eine übermäßige Menge an Prozessen. Wird diese Menge in einem gewissen Rahmen gehalten, so kann der CFS-Scheduler das  completely-fair -Prinzip gut erreichen. 
