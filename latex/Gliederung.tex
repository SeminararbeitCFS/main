Entwurf für Gliederung
----------------------

0. Abstract
1. Einleitung
	CFS Ablösung von O(1) seit Kernel-Version 2.6.23

	Literatur:
	Ingo Molnar, This is the CFS
		http://people.redhat.com/mingo/cfs-scheduler/sched-design-CFS.txt

	Inside the Linux 2.6 CFS - IBM DeveloperWorks
		https://www.ibm.com/developerworks/library/l-completely-fair-scheduler/

2. Theorie
2.1 kurze Wiederholung Scheduling
	Definition
	Ziele,Kriterien, die ein "guter" Scheduler Erfüllen muss

	Literatur:
	Pinedo, Michael: Scheduling: theory, algorithms, and systems. 4th ed.
	Springer, 2012
	Silberschatz, Ari: Operating System Concepts. Wiley, 2009

2.2 Theorethische Grundlagen ROT-SCHWARZ-Baum
	O-Aufwände, Einfügen, Suchen, Herausnehmen

	Literatur:
	Thomas H. Cormen, Charles E. Leiserson, Ronald L. Rivest, Clifford Stein: Introduction to Algorithms. 2. Auflage. MIT Press, Cambridge (Massachusetts) 2001, ISBN 0-262-03293-7, S. 273–301.

	Kurt Mehlhorn: Datenstrukturen und effiziente Algorithmen. Teubner Stuttgart 1988, ISBN 3-519-12255-3.

3. Der Completely-Fair-Scheduler
  	Grundlegender Aufbau / Internas
	Elementar: virtual runtime
	Priorities (wie werden Sie genutzt)
	Group Scheduling (Erweiterung)
	Scheduling-Classes
	Vergleich zum Vorgänger O(1) (hinsichtlich O-Notation,Arbeitsweise)
	Darstellung der C-Strukturen in Verbindung mit dem Rot-Schwarz-Baum
	C-Strukturen
	Methoden zum Einfügen, Suchen, Löschen von Blättern -> graphische Simulation?
	Tuning?

	Literatur:

	The CFS - Thomas Gleixner - Linuxfoundation Japan Symposium
		https://www.linuxfoundation.jp/jp_uploads/seminar20080709/lfjp2008.pdf

	Linuxjournal
		http://www.linuxjournal.com/magazine/completely-fair-scheduler

	The Linux Scheduler: a Decade of Wasted Cores - Paper
		https://www.ece.ubc.ca/~sasha/papers/eurosys16-final29.pdf

	Understanding the Linux kernel : [from I/0 ports to process management; covers version 2.6] / Daniel P. Bovet and Marco Cesati
		https://www.ub-katalog.fernuni-hagen.de/F/VXM2XQPLKBSBHETKB2VX56JV8NREJVL2PEYEJDBS3V6NVLGY8F-14084?func=full-set-set&set_number=003184&set_entry=000007&format=999

	Linux Kernel Development - A thorough guide to the design and implementation of the Linux kernel
		http://www.pi6.fernuni-hagen.de/downloads/OS/Scheduling/l-lkd-10.pdf

4. Zusammenfassung
