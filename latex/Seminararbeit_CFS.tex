% !TEX encoding = UTF-8 Unicode

% Beispiel für ein LaTeX-Dokument im Format "seminarvorlage"
\documentclass[ngerman]{seminarvorlage}
% ngerman = Deutsch in neuer Rechtschreibung, alternativ english

\usepackage[utf8]{inputenc} % Kodierung der Umlaute
\usepackage{babel} % automatische Sprachanpassung, Sprache siehe oben
\usepackage{cleveref} % für bequeme Referenzen, siehe \cref unten
\usepackage{amsmath}
\usepackage{amsfonts}
\usepackage{amssymb}
\usepackage{graphicx}

\begin{document}

% Unbedingt angeben: Titel, Autoren, E-Mail
% Freiwillig: Adresse
\title{The Completely Fair Scheduler}
\numberofauthors{2}
\author{
  \alignauthor Lukas Essig\\
    \email{lukas.essig@studium.fernuni-hagen.de}
  \alignauthor Peter Müller\\
    \email{peter.müller@studium.fernuni-hagen.de}
}

\maketitle

\abstract{Die Geschichte der Gummi\-bär\-chen ist voller
Über\-raschun\-gen\ldots} % Trennhilfe \- manchmal nützlich

\keywords{Linux, CFS, Rot-Schwarz-Baum, Scheduling, completely fair}

% Section-Überschriften werden in GROSSBUCHSTABEN umgestellt
\section{Einleitung}

Das Gummibärchen war schon immer ein Quell der Freude für Jung
und Alt, vgl. \cite{mingo,mjones}.

\section{Grundlagen zum Scheduling}
\subsection{Scheduling}\label{s:scheduling}
In der Fachliteratur wird der Begriff Scheduling mit zwar unterschiedlichen Begriffen, jedoch gleicher Bedeutung definiert. Michael Pinedo definiert Scheduling in seinem Werk \cite{mpinedo} folgendermaßen:
\begin{quote}
\textit{"`Scheduling is a decision-making process [...]. It deals with the allocation of resources to tasks over given time periods and its goal is to optimize one ore more objectives."'}
\end{quote}
So definiert er Scheduling als einen Entscheidungsprozess, der sich mit der zeitlichen Zuteilungen von Re"-ssourcen zu Aufgaben beschäftigt. Dabei ist das Ziel, ein oder mehrere Eigenschaften zu optimieren. Ferner erklärt er, dass die Ressourcen und Aufgaben in einer Organisation unterschiedliche Formen annehmen können. Ressourcen können beispielsweise Maschinen in einer Fertigungsanlage, Landebahnen auf einem Flughafen oder Verarbeitungseinheiten in einem Rechnersystem sein. Die Aufgaben wären analog Operationen in der Fertigunsanlage, Start und Landung auf dem Flughafen oder Programmausführungen im Computer. Eigenschaften der einzelnen Aufgaben sind zum Beispiel gewisse Prioritäten, frühester Startzeitpunkt oder ein Ablaufdatum. Die zu optimierende Ziele können ebenfalls viele verschiedene Formen annehmen. Das könnten die Reduzierung der gesamten Ausführungszeit einer Aufgabe sein oder die Minimierung der Anzahl von Aufgaben, die nach ihrer Fälligkeit abgeschlossen wurden.\\

Eine andere Definition ist von Alessandro Agnetis aus seinem Buch \cite{aagnetis}:
\begin{quote}
\textit{"`[...] by scheduling we mean all actions that have to be done in order to determine when each activity of a set is to start and to complete."'}
\end{quote}
So umfasst seiner Ansicht nach Scheduling diejenigen Tätigkeiten, die zur Bestimmung der Start- und Endzeit einer Aktivität aus einem Set getätigt werden müssen. Den Begriff Job spezifiziert er als Aktivität aus einem Set. So ist jeder Job im Wettkampf mit den anderen um die Nutzung von Zeit und Ressourcen. Als Ressourcen charakterisiert der Autor alles, was zur erfolgreichen Ausführung der Jobs erforderlich ist. Wie M. Pinedo benennt er Scheduling ebenso als einen Prozess, der sich mit der Zuteilung von Ressourcen zu Jobs beschäftigt. Seiner Meinung nach ist ein \textit{Schedule} (Plan) somit durch eine Menge von Startzeiten und zugeteilten Ressourcen bestimmt, der einige vordefinierte Anforderungen erfüllt.\\
Ein Problem, in welchem bei gegebenen Eingabedaten ein Plan gefunden werden muss, wird \textit{Scheduling-Problem} genannt (siehe \ref{s:probleme}). Ein \textit{feasible Schedule} (ausführbar) ist ein Plan, der alle Anforderungen des gegebenen Scheduling-Problems erfüllt \cite{aagnetis}.\\

Im Kontext des \textit{Prozess-Scheduling} erklärt Avi Silberschatz in \cite{asilberschatz}, dass dieses dafür verantwortlich ist, dass die Prozesse aus den Warteschlangen ausgeführt werden können. Dazu erteilt der Scheduler den Prozessen eine gewisse Zeit lang die CPU, in der sie diese Ressource benutzen können. Weitere Aufgaben des Prozess-Scheduling sind in Kapitel \ref{s:schedInformatik} erläutert.\\

Zusammenfassend kann also festgestellt werden, dass ein Scheduler Ressourcen und Aufgaben verwaltet. Dazu wird ein optimaler Plan erstellt, der sicherstellen soll, dass alle Aufgaben mit ihren Eigenschaften (Startzeit, Ablaufdatum, Priorität) und den benötigten Ressourcen erfolgreich ausgeführt werden. Oftmals ist es nicht einfach, einen optimalen Plan für die gegebenen Schedule-Anforderungen zu finden. Kriterien, welche durch den Scheduler optimiert werden sollen, sind in Abschnitt TODO vorgestellt.\\
\cite{mpinedo} und \cite{aagnetis} zeigen ein Beispiel, in dem die Rolle des Scheduling in einer realen Umgebung illustriert wird:
\\
\begin{description}
\item[Gate-Zuweisung am Flughafen]
An einem Airline-Terminal eines größeren Flughafens gibt es dutzende Gates und hunderte von Flugzeugen, die täglich starten und landen. Sowohl die Gates als auch die Flugzeuge sind nicht identisch. Manche Gates haben räumlich viel Platz, sodass hier auch größere Flugzeuge angekoppelt werden können. Andere sind so gelegen, dass es schwierig ist die Flugzeuge ohne Unterstützung anzukoppeln. Obwohl die Flugzeuge nach einem gewissen Plan ankommen und wieder abheben, gibt es doch zufällige Änderungen, die vom Wetter oder anderen nicht vorhersehbaren Ereignissen beeinflusst werden. Während ein Flugzeug an einem Gate steht, müssen die Passagiere die Maschine verlassen. Diese muss getankt, gereinigt und beladen werden und anschließend die Passagiere des nächsten Fluges einsteigen. Die Abflugzeit ist hier das Ablaufdatum, da bis zu diesem Zeitpunkt alle Aufgaben abgeschlossen sein müssen. Wenn jedoch absehbar ist, dass das Flugzeug nicht am Zielflughafen landen kann, wird es auch nicht abheben. Aus Vorschriftsgründen bleiben die Passagiere dann im Terminal anstatt im Flugzeug. Das Flugzeug würde für eine erweiterte Zeit am Gate stehen bleiben und blockiert es somit für andere.\\
Der Scheduler muss die Flugzeuge so den Gates zuweisen, dass dies physisch möglich ist. Dies impliziert, dass die Flugzeuge an die Gates passen und diese zur geplanten Ankunftszeit auch zur Verfügung stehen. Anforderungen wie die zeitliche Reduktion eines Arbeitsvorgangs für das Airline Personal oder die Reduzierung von Verspätungen müssen dabei optimiert werden.\\
In diesem Beispiel sind die Gates die Ressourcen und die Wartung der Flugzeuge die Aufgaben. 

%\item[Halbleiter Herstellung]
%Halbleiter werden in hoch spezialisierten Fabriken hergestellt. Meistens besteht die Herstellung aus vier Phasen: Wafer-Herstellung, Wafer-Untersuchung, Montage und finaler Test, wobei die Herstellung die komplexeste Phase ist. Die Wafer werden aus Metall- und Siliziumschichten hergestellt und jede Schicht muss einer Anzahl an Arbeitsgängen unterliegen. Manche Maschinen müssen extra für bestimmte Aufgaben vorbereitet werden. Die Anzahl an Bestellungen im Produktionsprozess liegt oft im Bereich mehrerer hundert Stück und jede Bestellung hat ihr eigenes Release- und bestätigtes Lieferdatum. Die Bestellungen werden in Jobs umgewandelt und mit Ablaufdaten verknüpft. Die Ausführung dieser Jobs kann zeitweise verspätet sein, wenn spezielle Maschinen belegt sind oder Jobs mit höherer Priorität den Aktuellen verdrängen. Unvorhersehbare Ereignisse, wie Maschinenausfälle oder eine länger als erwartete Herstellung, müssen ebenso berücksichtigt werden, da sie einen großen Einfluss auf den Schedule haben.\\
%Die Aufgabe des Schedulers umfasst die Einhaltung der maximalen Anzahl an Lieferterminen bei zeitgleicher Optimierung des Durchsatzes. Letzeres Ziel ist erfüllt, wenn die Geräteauslastung, unter anderem die der Engpassmaschinen maximiert ist. Zusätzlich müssen Ruhe- und Setupzeiten minimiert werden.
\end{description}
\begin{figure}[h]
	\centering
	\includegraphics[width=0.45\textwidth]{pictures/info_flow_service_system}
	\caption{Informationsfluss Service-System }
	\label{f:info_flow}
\end{figure}

Diagramm \ref{f:info_flow} unterstreicht die Aussage, dass Scheduling-Systeme viele Abhängig"-keiten zu anderen Infrastrukturen besitzen können bzw. müssen. Das Diagramm zeigt den Informationsfluss in einer Service-Organisation, beispielsweise eines Autoverleihs. Dabei wird das Scheduling von Einflussgrößen wie den Kunden, der Datenbank oder auch Vorhersage-Modulen geprägt. So gehören neben den Punkten aus den oben genannten Definitionen noch folgende Aufgaben zum Scheduling:
\begin{description}
\item[planning:] Planung der Aufgaben für die einzelnen Ressourcen
\item[dispatch:] dynamische Zuteilung einer Ressource zu einer Instanz (Job)
\item[processing:] Ausführung der Instanz
\item[control:] begleitende Auftragssteuerung
\item[monitoring:] Beobachtung des Zustandes
\item[feedback:] Rückmeldung von Ergebnissen und Abschlüssen aller verketteten Teilprozesse einer Instanz
\end{description}



Zusammenfassend ergeben die Definitionen aus \cite{mpinedo} und \cite{aagnetis} einen Prozess, der die Planung und Zuteilungen von Ressourcen zu Aufgaben sowie deren Ausführung und Beobachtung beschreibt. Dazu kommt die Einbeziehung sowie Auflösung von Abhängig"-keiten zu anderen Systemen, die Start- und Endzeit einer jeden Aufgabe beeinflussen.


	
\subsection{Rot-Schwarz-Baum}\label{s:rb_tree}
Der Rot-Schwarz-Baum stellt eine Erweiterung des bin-ären Suchbaums dar. Er wurde 1972 zum ersten mal von Rudolf Bayer unter dem Namen  \textit{symmetric binary B-Tree} vorgestellt.

Ein binärer Suchbaum eignet sich zum schnellen Finden von Schlüsselelementen. Dabei werden die Elemente nicht wie in einer Liste sequentiell durchsucht, sondern es wird der Baum in einer bestimmten Logik traversiert.
Damit kann im Vergleich zur sequentiellen Liste die Laufzeit für verschiedene Operationen verkürzt werden.

Die Anordnung der Elemente wird, wie im Namen bereits enthalten, als Baum vorgenommen. Dabei müssen bestimmte Kriterien für die Anordnung eingehalten werden. Liegt eine bestimmte Anzahl von Elementen vor, die als Binärbaum angeordnet werden sollen, so wird zunächst ein beliebiges Element ausgewählt und als Wurzel gesetzt. Damit kann es allerdings passieren, dass der Baum später nicht ausgeglichen ist und alle Äste unterschiedliche Dimensionen aufweisen. Dies würde zu einer unausgewogenen Suche führen. Daher ist es sinnvoll an dieser Stelle ein Element mit einem Schlüsselwert zu finden, welcher sich relativ in der Mitte der vorkommenden Werteskala und sich zudem auch bezüglich der Anzahl der Elemente in der Mitte befindet.

Danach werden die anderen Elemente der Reihe nach mit der Wurzel verglichen. Ist das momentan betrachtete Elemente größer als das Wurzelelement, so wird es an der rechten Seite unter der Wurzel angeordnet. Ist es kleiner als die Wurzel, so kommt es auf die linke Seite. Nun kann das nächste Element mit der Wurzel verglichen werden. Wie im obigen Fall wird es, wenn es größer ist, der rechten Seite zugeordnet und wenn es kleiner ist, der linke Seite. An dieser Stelle kann es jetzt passieren, dass auf der ausgewählten Seite bereits ein Blatt beziehungsweise ein Knoten existiert. Hier muss nun ein weiterer Vergleich stattfinden. Wiederum wird geprüft, ob das momentan ausgewählte Element größer oder kleiner ist als dieser Knoten. Dann wird es entsprechend unter dieses Element angefügt. Somit entsteht ein Baum. Wichtig ist in diesem Verfahren, dass jeder Knoten nur maximal zwei Kinder besitzen darf. 

Die Suche nachher im Baum gestaltet sich ähnlich. Soll ein Element im Baum gesucht werden, so wird ein Vergleich an der Wurzel des Baumes gestartet. Ist der Schlüs\-sel des zu suchenden Elements größer, so wird auf die nächste untere Ebene der rechten Seite gewechselt. Ist der Schlüssel kleiner, dann findet der Wechsel entsprechend auf der linken Seite statt. 

Der größte Vorteil dieser Anordnung besteht in der erheblichen Verkürzung der Laufzeit gegenüber beispielsweise einem Listenverfahreren. 
Als Vergleich soll die Laufzeit beim Suchen dienen.
Während beim sequentiellen Suchen in Listen der Maxmimalwert der Laufzeit $O(n)$ betragen kann ($n$ Anzahl vorhandener Elemente), so wird die Laufzeit beim binären Suchbaum auf maximal $O(h)$ verkürzt, wobei $h$ die Höhe des Baumes ist \cite{tcormen}.

Allerdings kann es auch durch ständiges Löschen und Ein-fügen von Elementen oder aber auch durch eine unvorteilhafte Wahl des Wurzelknotens dazu kommen, dass sich der Baum in einem Ungleichgewicht befindet. Dies bedeutet, dass eine Seite des Baumes erheblich mehr Knoten besitzt als die andere. Dadurch wird sich die Suchlaufzeit auf der einen Seite erhöhen und auf der anderen Seite ver\-kürzen. Im Extremfall wären alle Elemente auf einer Seite hintereinander angeordnet und somit wäre die Laufzeit gleich der einer sequentiellen Liste.

Eine solche Fehlanpassung kann mit Hilfe der Regeln eines Rot-Schwarz-Baumes kompensiert werden. Der Rot-Schwarz-Baum erweitert die Logiken eines binären Such\-baums. Damit ist es mög\-lich, einen sich ständig ändernden Baum in einem an\-nä\-hern\-den Gleichgewicht zu halten. 

Der Rot-Schwarz-Baum erhält seinen Namen durch ein zusätzliches At\-tri\-but, welches an jeden Knoten an\-ge\-hängt wird und als Wert die Farbe Rot oder Schwarz enthalten kann. 
Mit Hilfe dieses Attributes kann dann der Baum nach bestimmten Regeln bearbeitet werden. Diese Regeln sorgen dafür, dass immer ein grundlegendes Gleichgewicht im Baum herrscht. 
In Abbildung \ref{fig:rbtree} ist der Aufbau eines Rot-Schwarz-Baumes zu sehen.

\begin{figure}[h]
	\centering
	\includegraphics[width=0.45\textwidth]{pictures/redblacktree.png}
%	\caption{RED-BLACK-TREE~\protect\cite{rtoal}}
	\caption{Rot-Schwarz-Baum}
	\label{fig:rbtree}
\end{figure}

 
Nach \cite{tcormen} müssen folgende Regeln eingehalten werden, damit ein Rot-Schwarz-Baum zustande kommt:

\begin{enumerate}
	\item Jeder Knoten ist entweder rot oder schwarz.
	\item Die Wurzel ist schwarz.
	\item Jedes Blatt (NIL) ist schwarz.
	\item Wenn ein Knoten rot ist, dann sind seine beiden Kinder schwarz.
	\item Für jeden Knoten enthalten alle einfachen Pfade, die an diesem Knoten starten und in einem Blatt des Teilbaumes dieses Knotens enden, die gleiche Anzahl schwarzer Knoten. 
\end{enumerate}

Wie bereits Thomas H. Cormen in \cite{tcormen} beweist, ergeben sich die maximalen Laufzeiten für verschiedene Operation, wie zum Beispiel das Suchen im Baum, auf einen Höchstwert von $O(\log n)$. 
%kann ein Baum, welcher nach diesen Regeln erstellt wird, eine maximale Höhe von $2\log_2(n + 1) $ erreichen. Dem entsprechend ergeben sich dann die maximalen Laufzeiten für verschiedene Operation, wie zum Beispiel das Suchen im Baum, auf einen Höchstwert von $O(\log n)$.

Beim Einfügen oder Löschen in den Baum werden Verbindungen oder Knoten verändert. Dabei kann es passieren, dass der Baum wieder in ein Ungleichgewicht gerät. 
Um dieser Problematik entgegen zu wirken, werden spezielle Einfüge- beziehungsweise Lösch\-operationen angewendet, welche so konstruiert sind, dass immer die fünf oben genannten Regeln eingehalten werden. 

So müssen innerhalb dieser Einfüge- oder Lösch\-ope\-ra\-ti\-on\-en zum Beispiel Rotationen von Knoten unter bestimmten Bedingung durchgeführt oder die Farben beziehungsweise auch die Zeigerstruktur geändert werden.

Werden die fünf Regeln unter allen Bedingungen und zu jeder Zeit eingehalten, so kann immer die maximale Laufzeit garantiert werden und außerdem können der kleinste und der größte Schlüssel immer direkt auf der linkesten beziehungsweise auf der rechtesten Seite gefunden werden.

Der Rot-Schwarz-Baum hat im Vergleich zu anderen balancierten Baumvarianten eine niedrigere Operationzeit, da er z.B. weniger Rotationen nach Einfüge- bzw. Lösch\-vorgängen ausführen muss.
Daher ist er für Anwendungen mit einer hohen Zugriffsrate, wie z.B. dem CFS-Scheduler, zu bevorzugen.

%Mit den Regeln von RED-BLACK wird mit Hilfe von zum Beispiel Rotationen um das den Parent Knoten, immer für das richtige Gleichgewicht gesorgt.


\subsection{Das \glqq Completely-Fair\grqq{} Prinzip}\label{s:fair}
Die wichtigste Aufgabe eines Schedulers ist die Verteilung von knappen Ressourcen unter einer Menge von Klienten. 
Um dies zu realisieren, wurden bereits viele Anwendungen in Theorie und Praxis entwickelt.
Eine sehr bekannte Entwicklung beruht auf der Idee, jedem Klienten ein eigenes Gewicht zu zuweisen, welche dann je nach Höhe zum Gebrauch der begrenzten Ressource berechtigt.

Diese Verfahren, welches unter anderem unter dem Namen \glqq Proportional Share Scheduling\grqq{} bekannt sind, wurden bereits vor einigen Jahrzehnten entwickelt und fanden dann z.B. in einem gewichtetem Round-Robin Verfahren Anwendung. Auch unter UNIX fand dieses Verfahren schnell als Scheduler Anwendung mit Hilfe einer Priorität-Steuer\-ung. Dieses Verfahren ist schnell und benötigt immer eine konstante Zeit zum Auswählen des nächsten Klienten. 

Die Proportional-Scheduler können die Verteilung der Ressource hauptsächlich über zwei Wege erreichen:
\begin{enumerate}
	\item Über die Frequenz. Das heißt, wie oft bekommt ein Klient die Zuteilung. Dies kann zum Beispiel realisiert werden, mit einer entsprechenden Positionierung in einer Warteschlange.
	\item Über die Zeit. Dafür wird das Zeitquantum, welches einem Klienten zur Verfügung gestellt wird, entsprechend verlängert oder verkürzt.
\end{enumerate}

Wird die Scheduler-Logik mit Hilfe von Prioritäten gesteuert, entsteht schnell ein entscheidendes Problem. Die Klienten mit den höchsten Prioritäten werden bevorzugt und bei anderen Klienten mit niedrigeren Prioritäten kann der Fall auftreten, dass sie keine Zuteilung mehr bekommen und somit \glqq verhungern\grqq{}.
Um diese Problematik zu umgehen, wurde mit der Entwicklung alternativer Scheduling Algorithmen begonnen. Ein wichtiges Ziel sollte sein, eine gewisse Fairness unter den Klienten zu erreichen. Damit erhielten diese Verfahren den Namen \glqq Fair Scheduler\grqq{}. Mit Hilfe dieser Art von Schedulern soll das Problem der ungleichmäßigen bzw. ungerechten Verteilung der Prioritätssteuerung behoben werden.

Eine perfekte Fairness kann nur gebildet werden, wenn die Klienten bewertet werden und aufgrund dieser Bewertung ihre Zuteilung auf die Ressource erhalten.
Eine Möglichkeit für eine solche Bewertung ist ein Vergleich von verbrauchten Kontingent des Klienten zu Verbrauchten Kontingent aller aktiven Klienten in einem bestimmten Zeitraum. 
Dies wird nach \cite{usenix} wie folgt definiert:

\begin{equation}
W_{o,A}(t_1,t_2) = (t_2 - t_1) \frac{S_A}{\sum_i S_i}
\label{eq:perfect_fairness}
\end{equation}

Hierbei ist $S_A$ der proportionale Anteil des Klienten $A$, $S_i$ die Anteile aller aktiven Klienten und $W_{o,A}$ stellt als Ergebnis eine optimale proportionale Menge des Klienten in der Zeit zwischen $t1$ und $t2$ dar.

Im idealen System, wenn alle Klienten ihre Zuweisung gleichzeitig erhalten und ihren geforderten Ressourcenanspruch simultan verbrauchen könnten, wäre es möglich die obige Gleichung für alle Klienten entsprechend aufrecht zu erhalten.
Damit wäre eine \glqq completely-fair\grqq{} Verteilung gegeben. Allerdings ist dies in der Praxis nicht möglich, da immer nur ein Klienten für die Ressource die Zuteilung bekommen kann.
Um eine Bewertung der aktuellen angewendeten Strategie zur idealen \glqq completely-fair\grqq-Strategie zu erhalten, kann ein sogenannter \glqq service time error\grqq{} mit folgender Formel berechnet werden:

\begin{equation}
E_A(t_1,t_2) = W_{S,A}(t_1,t_2) - W_{o,A}(t_1,t_2)
\label{eq:error_fairness}
\end{equation}

Diese Differenz gibt den Abstand vom gegebenen Algorithmus bzw. der aktuellen Strategie zum optimalen Algorithmus an.
Ist der Abstand positiv, so bekommt der aktuelle Klient mehr Zuteilung als er mit dem idealen Algorithmus bekommen würde. Ist der Abstand negativ, so erhält der Klient im Vergleich weniger.

Je kleiner der Wert aus Formel \ref{eq:error_fairness}, desto mehr erreicht der aktuelle Algorithmus die Funktionsweise des optimalen \glqq completely-fair\grqq{}-Algorithmus.
%Ziel in einer praktischen Lösung im Bezug auf \glqq completely-fair\grqq{} soll es sein, diesen Abstand
 





\section{Der Completely-Fair-Scheduler}
\subsection{Funktionsweise}\label{s:cfs_fktweise}
Das Ziel eines  Completely-Fair-Schedulers{} soll es sein, eine gute Annäherung an das Verhalten eines perfekten Multitasking-Prozessor zu errreichen.

Ein perfekter Multi\-tasking-Prozessor würde jedem Prozess immer ein perfektes $1/n$ Verhältnis zuweisen, wobei $n$ die Menge aller Prozesse darstellt. 
Bei nur zwei laufenden Prozessen heißt das, dass beide Prozesse zeitgleich mit jeweils 50 Prozent der Prozessorressource arbeiten dürfen. Dies ist in der Praxis allerdings nicht möglich, da jeder Prozessor immer nur eine Aufgabe zu einem Zeitpunkt bearbeiten kann.

Unter Linux und UNIX-artigen System werden den beiden Prozessen jeweils Zeitscheiben zugeordnet, und es findet eine sequentielle Abarbeitung statt. In diesem Fall werden die Prozesse abwechselnd mit jeweils 100 Prozent Prozessorzuteilung abgearbeitet.

Um mit diesem Prinzip eine Annäherung an einen perfekten Multitasking-Prozessor zu erreichen, müsste man die Zuteilung für jeden Prozess in unendlich kleine Zeitscheiben zerlegen und alternierend dem Prozessor zuweisen. Würde man an dieser Stelle eine Messung der Zuteilung durchführen, sähe es tatsächlich so aus, als würde alle Prozesse ihre benötigte Zeit gleichmäßig verbrauchen.
Allerdings hat diese Methode eine erheblichen Nachteil. Ein einziger Umschalt\-vorgang be\-nötigt selbst z.B. durch Ein- und Auslagerung der verschieden Daten im Speicher, wieder einen großen Teil der Prozessorleistung. Dadurch würde ein sehr schlechtes Verwaltungs- zu Verarbeitungsverhältnis entstehen und damit wäre der Scheduler sehr uneffizient.

Mit dem CF-Scheduler muss nun eine Umgebung geschaffen werden, welche stark an der des idealen Multitasking-Prozessor anlehnt und wiederum aber auch eine Effiziente Methode zur Ausnutzung der Prozessorzuteilung anbietet.
Anders als bei den prioritätsbestimmten Scheduler, welche über Zeitscheiben und Warteschlangen die Prioritäten verwalten, berechnet der CF-Scheduler die Zuteilung im Bezug zu allen anderen Prozessen welche sich im Wartemodus befinden. Der  \textit{nice}-Wert, welcher in bereits in älteren Scheduler\-ver\-fahren für die Erstellung der Zeitscheiben zuständig war, wird jetzt verwendet, um den Prozessen ein Gewicht zuzuordnen. Ist der  \textit{nice}-Wert hoch, was eine niedrige Priorität bedeutet, so bekommt der Prozess auch ein niedriges Gewicht zugeteilt. Andersherum bekommt der Prozess ein hohes Gewicht, wenn der  \textit{nice}-Wert niedrig ist.

Mit Hilfe dieses Gewichtes wird dann das  Completely-Fair-Prinzip [s. Kapitel \ref{s:fair}] angewendet. Jeder Prozess erhält eine Zeitscheibe, welche proportional zum eigenen Gewicht im Verhältnis zu den Gewichten aller Prozesse ist. Die Wahl des nächstes Prozesses fällt immer auf den Prozess, welcher bis zu diesem Zeitpunkt die geringste Zuteilung des Prozessors erhalten hat.
Zur Berechnung einer Zeitscheibe, setzt der CF-Scheduler eine sogenannte „targeted latency“. Diese stellt einen Richtwert für die verfügbare Zeit aller anstehenden Prozesse dar, welcher äquivalent im perfekten Multitasking erreicht würde.

Wenn zum Beispiel eine „targeted latency“ von 20ms errechnet wird und zwei Prozesse mit dem gleichen  \textit{nice}-Wert in der Warteschlange stehen, wird jedem Prozess eine Zuteilung von 10ms gewährt.

Ein Problem ergibt sich bei dieser Anwendung, wenn die Anzahl der wartenden Prozesse gegen unendlich läuft. Damit würde die zugewiesene Zeit gegen Null laufen und es würde kein Prozess mehr die Berechtigung einer Zuteilung erhalten. Um dieses Problem zu umgehen, wird eine minimale Granulität eingeführt ,welche normalerweise mit einer Millisekunde gewählt wird. Das heißt jeder Prozess erhält immer mindestens eine Zuteilung von einer Millisekunde, egal wie groß die Menge der anstehenden Prozesse zu diesem Zeitpunkt ist. Dadurch wird allerdings das „faire“ Verhalten des Schedulers beeinträchtigt. 

Auch findet eine weitere Unterscheidung zur Nutzung der  \textit{nice}-Wert statt. Wurde zuvor der absolute  \textit{nice}-Wert für eine Prioritätsermittlung benutzt, so spielt jetzt nur noch der relative Abstand zum nächsten  \textit{nice}-Wert eine Rolle.

Ein passendes Beispiel ist in \cite{rlove} illustriert. Gegeben ist ein Zustand mit zwei Prozessen. Ein Prozess hat den  \textit{nice}-Wert 0, der andere den  \textit{nice}-Wert 5. Damit würde sich eine Prozessorzuteilzeit von 15ms für den Prozess mit  \textit{nice}-Wert 0 und eine Zuteilzeit von 5ms für den Prozess mit  \textit{nice}-Wert 5 ergeben.
Im Vergleich dazu hätte man einen weiteren Zustand mit wiederum 2 Prozessen, aber dieses mal mit den  \textit{nice}-Werten 10 und 15. Hier würde das selbe Ergebnis mit 15ms und 5ms zustande kommen. Damit wird gezeigt, dass die Höhe des  \textit{nice}-Wertes nur noch die geometrische Differenz ändert.

Anhand dieser Erläuterung kann gesehen werden, dass der CF-Scheduler nur eine Annäherung an ein perfektes faires Scheduling erreicht. Das große Problem der CF-Scheduler ist eine übermäßige Menge an Prozessen. Wird diese Menge in einem gewissen Rahmen gehalten, so kann der CF-Scheduler das  completely-fair -Prinzip gut erreichen. 


@lukas: erweiterung scheduling-gruppen nachschlagen
\subsection{Vergleich zum O(1)-Scheduler}
@lukas
\subsection{C-Strukturen}
@lukas
\section{Zusammenfassung}

% Eine neue Spalte anfangen mit
\pagebreak

% Bibliographie entweder direkt hier eingeben (nur im Notfall)...
%\begin{thebibliography}{9}
%\bibitem{acmcategories}
%How to classify works using ACM's computing classification system.
%\newblock \url{http://www.acm.org/class/how_to_use.html}.
%
%\bibitem{Ivory2001}
%M.~Y. Ivory and M.~A. Hearst.
%\newblock The state of the art in automating usability evaluation of user
%  interfaces.
%\newblock {\em ACM Comput. Surv.}, 33(4):470--516, 2001.
%
%\end{thebibliography}

% ... oder die Bibliographie mit Hilfe von BibTeX generieren,
% dies ist auf jeden Fall die bessere Lösung und sollte nach
% Möglichkeit immer verwendet werden:
\bibliographystyle{abbrv}
\bibliography{literatur} % Daten aus der Datei literatur.bib verwenden.

\end{document}
