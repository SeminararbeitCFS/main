% !TEX encoding = UTF-8 Unicode

% Beispiel für ein LaTeX-Dokument im Format "seminarvorlage"
\documentclass[ngerman]{seminarvorlage}
% ngerman = Deutsch in neuer Rechtschreibung, alternativ english

\usepackage[utf8]{inputenc} % Kodierung der Umlaute
\usepackage{babel} % automatische Sprachanpassung, Sprache siehe oben
\usepackage{cleveref} % für bequeme Referenzen, siehe \cref unten
\usepackage{amsmath}
\usepackage{amsfonts}
\usepackage{amssymb}

\begin{document}

% Unbedingt angeben: Titel, Autoren, E-Mail
% Freiwillig: Adresse
\title{The Completely Fair Scheduler}
\numberofauthors{2}
\author{
  \alignauthor Lukas Essig\\
    %\affaddr{Goldstraße 23}\\
    %\affaddr{12345 Musterstadt}\\
    \email{lukas.essig@studium.fernuni-hagen.de}
  \alignauthor Peter Müller\\
    \email{peter.müller@studium.fernuni-hagen.de}
}

\maketitle

\abstract{Die Geschichte der Gummi\-bär\-chen ist voller
Über\-raschun\-gen\ldots} % Trennhilfe \- manchmal nützlich

\keywords{Süßigkeiten, Gelatine, Bär, Lebensmittelfarbe.}

% Section-Überschriften werden in GROSSBUCHSTABEN umgestellt
\section{Einleitung}

Das Gummibärchen war schon immer ein Quell der Freude für Jung
und Alt, vgl. \cite{mingo,mjones,Ivory2001,Black1988}.

\section{Theorie}

\subsection{Scheduling}

\subsection{ROT-SCHWARZ-Baum}

\section{Der Completely-Fair-Scheduler}

% Eine neue Spalte anfangen mit
\pagebreak

% Bibliographie entweder direkt hier eingeben (nur im Notfall)...
%\begin{thebibliography}{9}
%\bibitem{acmcategories}
%How to classify works using ACM's computing classification system.
%\newblock \url{http://www.acm.org/class/how_to_use.html}.
%
%\bibitem{Ivory2001}
%M.~Y. Ivory and M.~A. Hearst.
%\newblock The state of the art in automating usability evaluation of user
%  interfaces.
%\newblock {\em ACM Comput. Surv.}, 33(4):470--516, 2001.
%
%\end{thebibliography}

% ... oder die Bibliographie mit Hilfe von BibTeX generieren,
% dies ist auf jeden Fall die bessere Lösung und sollte nach
% Möglichkeit immer verwendet werden:
\bibliographystyle{abbrv}
\bibliography{literatur} % Daten aus der Datei literatur.bib verwenden.

\end{document}
